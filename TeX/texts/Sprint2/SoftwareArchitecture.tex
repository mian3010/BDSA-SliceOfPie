\subsection{Software Architecture}
\subsubsection{Model-View-Controller(offline)}
To ensure a great level of extensibility and reusability, we implement the GUI, the controller and the model as a model-view-controller pattern. The view our winforms application. The responsibility of the view is:
\begin{itemize}
\item  To receive input from the user and publish this input as events. 
\item To listen for changes in the model. 
\end{itemize}
We try to keep the logic in the GUI at an absolute minimum, so that adding a new GUI or an additional GUI for the program becomes an easy operation, that requires minimum knowledge about how the rest of the program logic  works. The main part of our controller module is the controller, that is auto-generated by the WinForms framework. The controller module is responsible for listening to the events from the GUI and for calling the right methods on the model. The controller does not contain any data and does not need to know anything about the actual implementation of the model, except the interface contracts. The model module consists of several classes, but the main class is the IAdministrator. The class is responsible for communicating with the persistence framework, communicating with the networking part of the program, synchronizing files and publishing events related to changes in the model. It acts as a facade to the entire back-end part of the offline program. The view and the controller does not know about anything else but the IAdministrator.
\newline
\subsubsection{S2.4.2 Quality scenarios}
During development of Slice Of Pie we have created some quality scenarios that the system should fulfill in a full scale program. Though this is a proof of concept project we consider these requirements as important goals during the development process as we use them as one of many tools to help us obtain the best possible solutions for our project.
The following quality scenarios is a list of measurable requirements for further use in an architectural factor table:
\begin{itemize}
\item Reliability -  When connection to the server is lost all changes from database should be recorded in a local log and synchronization should continue within 2 seconds when the connection to the server is reestablished.
\item Reliability - When connection to the server is lost all changes on local files should be saved locally and synchronized to the server within 2 seconds as soon as the connection is reestablished.
\item Performance - Slice Of Pie should have 90\% availibility during a period of one year including downtime for  maintenance.
\item Performance - When Slice Of Pie system is launched, a connection to the server is established and a synchronization button is pushed a maximum of 2 seconds delay under normal conditions before synchronizing from the database begins.
\end{itemize}