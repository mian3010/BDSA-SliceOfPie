\subsection{Scrum Planning}
\subsubsection{Sprint Planning}
After the first sprint finishes, we move on to planning the next sprint. We decide that the result of sprint 2 should be a finished, documented and tested prototype. Additionally, we also want a report with most of its contents finished, ready for retouching and corrections. We want Sprint 3 to emulate a release-sprint, aiming at putting the final touches to the report(grammar, fonts etc.), organizing the OOAD artifacts, deploying the prototype and practical matters, such as printing and burning CDs. Therefore we add all the backlog items to the sprint, that we consider a requirement for the proof of concept to be fulfilled. [reference: sprint backlog 2]\\
% Add ref
We give the implementation tasks a high priority. Ideally we want each sprint to consist of analysis, design, implementation and deployment, but we consider it important to implement key features of the program before we elaborate more on the system requirements.\\
\subsubsection{Daily Scrum}
According to the Scrum methodology, each full workday (in all sprints) is initiated with a daily Scrum. In this 15 minute meeting, we each report how progress is made with each task that is on the product backlog. Moreover, each team member can request help from other member in regards to difficult tasks. We use the Task Board of our Scrum tool as our primary form of task management. Each team member is standing up as to signal the immediate focus needed to do an effective meeting. In any case where a team member need some help, the Scrum Master either immediately facilitate the help needed (e.g declaring another members to a particular task) or he undertakes the responsibility to get the help needed later. Below is a picture a of a Scrum meeting with all members standing up around the Task Board, which is projected on a canvas.\\
%[insert picture of daily scrum, shrink it down bra].
After the Daily Scrum, the Scrum Master reviews the burndown chart for the sprint and compares it to the release date. Included below is a screenshot of a burndown chart as it looks in our Scrum Tool (this picture is from Sprint 1, though):\\
% Add pic
The final scrum activities we’ll discuss is the inspect and adapt related meetings at the end of a sprint. However, in order to maintain our chronological order of the report, we’ll defer the discussion of additional scrum-related activities such as: Scrum Retrospective and Scrum Review to the end of this sprint documentation.\\
\newpage