\subsection{Scrum Revisited}
To mark the end of our longest and most code-intensive sprint, we’ll discuss each of the inspect and adapt activities done at the end of a sprint: the Scrum Retrospective and the Scrum review.
\subsubsection{Review}
Both the retrospective and the review of the second sprint is probably the most relevant ones of the sprints, as the product is at the final stages and the third sprint will not contain much of a review. In addition, we’ll review the tasks related to the report. \\
We place importance on the review in Sprint 2 for several reasons:
\begin{itemize}
\item First, it allows the main developers on the GUI’s to present their work. Process-oriented errors is easily revealed in the review when the present clicks around in the GUI etc.
\item A complete overview of the documentation as it currently is is reviewed and each related artifact is presented. This also reveals outdated artifacts.
\item Last, the Scrum Master compares the goals of the sprint with that of the achieved goals. 
\end{itemize}
The overall goal of the second sprint, to complete our core product catalog and the main body of our report is reviewed as successful. Our review lists the following features in our program:
\begin{itemize}
\item A fully functioning online GUI made in Razor ASP.NET. The GUI can create, edit and save documents. Moreover, the GUI can view a list of meta data related to the document as well as a change history of the document.
\item A functioning offline GUI. The GUI can create and edit documents, but does not support showing images in the editor. Additionally, the system saves documents and its changes to the disc.
\item A well functioning synchronization process. The offline GUI can request synchronization with the remote storage based on a user email. The model will then download any files not on the disk already and display them in the GUI. Additionally, changes made to a document online will be displayed offline as well. However, the system does not handle conflicts except mark them in the system.
\end{itemize}
\subsubsection{Retrospective}
As our Scrum team is dissolving at the end of the project, we do the activity as an academic exercise and to improve any new teams we might participate in. The following items were noted and discussed at the Retrospective:
\begin{itemize}
\item The number of sprints we did in our project period was too many. Doing a sprint with a length less than a week makes it very hard to do serious initiation and inspection activities, as they might take up more time than they might streamline during the sprint.
\item Backlog items must in general be very specific and well-defined. If a backlog item is too general, it will create estimation problems. The way to resolve this is either to split the item up during a Sprint planning meeting before any team member commits to the task (recommended) - or split the item into several sub tasks immediately upon marking the item as `In Progress'.
\item Scrum Mastering is a very useful practice and can be upgraded to increase the team’s effectiveness greatly. In an ideal situation, a Scrum Master spends most of his time facilitating the needs of the Team. In general, having a member of the team helping the other members with a fresh set of eyes in case of debugging, report writing help et cetera is something that can facilitate almost all the problems we have encountered during the project.
\end{itemize}
\newpage