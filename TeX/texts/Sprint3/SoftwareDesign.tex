\subsection{Software Design}
In sprint 3, most of our design consists of testing, fixing bugs and solving logical errors. We will not implement any new features and make any drastic changes to the existing system design. A lot of work also lies in the documentation of the existing code. Both the documentation in the actual classes, as well as the final diagrams of our system. 
\subsubsection{Case: Synchronization}
One core feature that has reached its final form in sprint 3 is the synchronization process. Below we have a full sequense diagram describing the synchronization, reaching from the offline GUI to the Server:\\
[Appendix, Figur \ref{clientsyncsequencediagram} and \ref{serversyncsequencediagram}, page \pageref{clientsyncsequencediagram} and \pageref{serversyncsequencediagram}]\\
The diagram is autogenerated from Visual Studio. 

% From Sprint 2:
%Below is an example of an UML communication diagram made to support a process as described in a use case, U#2. Please note that the diagram illustrates the design as we thought it out at the start of the sprint. A section regarding the diagrams used for ‘real’ code documentation is presented in Sprint 3. [reference CD5A and CD5B]
