\section{Sprint 1: First iteration}
In this section, we will present and discuss the development that took place in the first iteration of our project. This includes a relevant presentation of the Scrum methodology as we have applied it in the sprint. We will mostly present artifacts created that are relevant to the inception and elaboration phases of the unified process.
\subsection{Initial Scrum}
Scrum is an iterative framework designed to manage software development projects [REFERENCE:\\ %http://www.scrum.org/Portals/0/Documents/Scrum%20Guides/Scrum_Guide.pdf, p. 3]. 
The methods originally intends to include a team of up to eight persons, along with a Scrum Master and a Product Owner. However, as we were only five persons we had no choice but to appoint these roles to team members. Moreover, we decided that we wanted a more democratic approach to the features that we included in our project; this implied that the responsibility of creating product backlog items were spread out on the whole group instead of relying solely on the Product Owner [scrum guide, p. 12].\\
\newline
Before initializing our first sprint, we decided to conduct two sprint planning meetings in somewhat extended versions. Since this were our first project in which we’ve used Scrum, we decided that we would ignore the time bounds that are specified in the Scrum guide [scrum guide, p. 9]. This allowed us to discuss and familiarize ourselves with the Scrum concepts.\\
The initial meetings were divided into several parts: First, we agreed on using the Scrumwise[ref: http://www.scrumwise.com] tool for organizing our Scrum activities. Then, we all democratically built the product backlog. \\
An important part of the formation of our team before the first sprint were also doing a formal written definition of done. This includes defining a strategy for testing, coding, documentation and acceptance from the rest of the team. The definition of done can be viewed in [section whatever: definition of done]\\
Lastly, we had to agree on the number and length of the sprints we were going to do. This also meant creating an overall timeline of the sprints, which can be viewed in APPENDIX A.1[sprinttimeline.jpeg]  These initial exercises allowed us to conduct the two, more formally defined meetings of sprint planning: part one of two, respectively. \\
A partial view of the product backlog can be found in APPENDIX A.2 [backlog1.jpeg]\\
\subsection{Sprint planning $|$ \& $\|$}
After our initial Scrum agreements, we conducted planning meeting part one for the next sprint. This involved several noteworthy items:
\begin{itemize}
\item Creating a capacity plan to get an overview of our initial work capacity [Appendix A.3: Capacity Plan.jpeg]
\item Estimating difficulty and priority of the items on the Product Backlog
\item And elaborating on the meanings of the different backlog items to each other (as part of not having a single product owner).
\end{itemize}
Then moving on to the second part of the meeting, where the team were:
\begin{itemize}
\item committing to highest prioritized tasks and estimating their time to complete
\item distributing the work so the workload was fair on each individual
\item initialize the sprint in the Scrumwise tool and review the burndown chart generated
\end{itemize}
An overview of the first Sprint Backlog is available in [Appendix A.4: Sprint backlog].\\
It is worth nothing that even though we decided on all having influence on the product backlog, the management of the backlog were deferred to one person, the “Product Owner”.\\

\subsection{Use Cases}
\begin{table*}[ht]\centering
  \ra{1.3}
  \begin{tabularx}{\textwidth}{@{}rXl@{}}\toprule
    \textbf{ID} & \textbf{Description} & \textbf{Priority} \\\hline
    1 & CreateDocumentOnline & High   \\\hline
    2 & SynchronizeChanges   & High   \\\hline
    3 & MultipleUserEdit     & High   \\\hline
    4 & ViewVersionHistory   & Medium \\
    \bottomrule
  \end{tabularx}
  \caption{Our use cases}
  \label{usecases}\centering% 
\end{table*}
\subsubsection{CreateDocumentOffline}
\textbf{Use Case \#1:} CreateDocumentOffline\\
\textbf{ID:} U\#1\\
\textbf{Primary Actor:} User / Writer\\
\textbf{Stakeholders:}\\
\underline{1. User:} Wants effective access to a bunch of documents. Wants to share these documents with to other users. Want to be able to access documents in his/her home computer, but still access them elsewhere through a web interface\\
\underline{2. Other users:} Potential other users, who wants to use the same document on their home computer or through the web interface.\\
\newline
\textbf{Precondition:} The user is authenticated into the system.\\
\textbf{Postcondition:} The document is created and saved locally.\\
\newline
\textbf{Main success scenario:} \\
1. User creates a new document on his local computer.\\
2. System saves the document offline and online. \\
3. User edits the newly created document to his liking. When he's done editing, he'll request a save from System.\\
4. System saves the changes made offline and online.\\
\indent\textit{Steps 3-4 are repeated until User is satisfied with the document.}\\
5. User quits the system repeats steps 1-6 until satisfied.\\
6. System commits changes offline and online, recording a session of editing to the history of the document.\\
\newline
\textbf{Extensions}\\
\textbf{2.} System has no online connection
\begin{enumerate}
\item System alerts User that his changes will not be saved online. System then defers synchronization to processes shown in U\#3.
\item User continues editing as in the main scenario until done.
\item The System makes an offline commit, recording a session of editing that differs from another, possible offline editing session.
\end{enumerate}
\subsubsection{SynchronizeChanges}
\textbf{Use Case \#2:} SynchronizeChanges\\
\textbf{ID:} U\#2\\
\textbf{Primary Actor:} User \\
\textbf{Stakeholder:}\\
\underline{1. User:} Wants his files to be available both offline and online. Being available offline permits the user to always access them and edit them even when there is no connection. When online, he can access them from all computers.\\
\newline
\textbf{Precondition:} The user is authenticated into the system.\\
\textbf{Postcondition:} The local folder is synchronized with the online folder\\
\newline
\textbf{Main success scenario:}\\
1. User opens the program\\
2. System synchronizes the local folder with the online folder.\\
3. User edits a document from the folder locally.\\
4. User saves the document.\\
5. System synchronizes the local folder with the online folder.\\
\newline
\textbf{Extension:}\\
\textbf{2.} System fails to synchronize with the online folder
\begin{enumerate}
\item User tries to synchronize using a button.
\item User repeat 2b until 2d.
\item System synchronizes with the online folder.
\end{enumerate}
\subsubsection{MultipleUserEdit}
\textbf{Use Case \#3:} MultipleUserEdit\\
\textbf{ID:} U\#3\\
\textbf{Primary Actor:} User1, User2\\
\textbf{Stakeholder:}\\
\underline{1. User:} Wants to edit in a document, that is shared with another user. Wants the document to be saved online and synchronized with both his and the other user(s) local document folders.\\
\newline
\textbf{Precondition:} User is authenticated. User has created a document as shown in U\#1.\\
\textbf{Postcondition:} The two documents are merged and saved online.\\
\newline
\textbf{Main success scenario:}\\
1. User1 shares his document with user2 online.\\
2. User2 synchronizes his local folder with the online folder.\\
3. User1 edits the document locally.\\
4. User1 synchronizes with the online document folder.\\
5. User2 edits the document locally.\\
6. User2 saves the document locally.\\
7. User2 synchronizes with the online folder.\\
8. System merges the online document with the changes made in user2´s document.\\
9. System presents the merged document and the original document to user2.\\
10. User2 accepts the merged document.\\
11. System saves the new merged document online.\\
\newline
\textbf{Extensions:}\\
\textbf{2.} User2 has no internet connection
\begin{enumerate}
\item User2 fails to synchronize with the system
\item User2 does not receive the file from the online folder
\end{enumerate}
\textbf{10.} User2 does not accept the merged document
\begin{enumerate}
\item User2 edits in the merged document
\item User2 accepts the edited, merged document.
\item System saves the new merged document online.
\end{enumerate}
\subsubsection{ViewVersionHistory}
\textbf{Use Case \#4:} ViewVersionHistory\\
\textbf{ID:} U\#4\\
\newline
\textbf{Primary Actor:} User\\
\textbf{Main:}\\
The user wants to view a history for a file in the Slice Of Pie system. He'll select a file from the graphical user interface in the system and click on a show version history button. The system will retrieve version history for the file and display it in a new window. \\
\textbf{Alternate:}\\
The user wants to revert document to an earlier state. This use case has yet to be defined and elaborated on.\\

\subsubsection{CreateDocumentOnline}
\textbf{Use Case \#5:} CreateDocumentOnline\\
\textbf{ID:} U\#5\\
\newline
\textbf{Primary Actor:} User\\
\textbf{Main:}\\
The user want to create a document using the Slice Of Pie web page. He'll enter the front page and log in to his account. A page with the file list will show up. The user will push the green plus button in the upper right corner. The creator page will show and the user can enter a filename, a document path, a document name and the the content of the document. The user will click on the save button which is an icon of a disc. He'll now see the Editor window with the content and the filename.\\
\newline
\textbf{Precondition:} The user have a running internet connection.\\

\subsubsection{EditDocumentOnline}
\textbf{Use Case \#6:} EditDocumentOnline\\
\textbf{ID:} U\#6\\
\newline
\textbf{Primary Actor:} User\\
\textbf{Main:}\\
The user will edit a document online. He'll click on the document he will edit. He can now see the document in the viewer window. He will click on the edit button. It is the button with the pencil image. He now enter the editor window and he can enter the content of the document. When he is done editing he now push the save button.\\
\newline
\tekstbf{Precondition:} The file list is not empty adn the user have a running internet connection.
\newpage
% Add figures
\section{Database Design}
We use a database on our server, to keep track of who owns which files, who has access and who made what changes to them.\\
Our relational database contains eight tables:\\
\textbf{User} describes a user. Email is used as primary key\\
\textbf{File} describes a file on the server. We use serverpath to describe the path to the folder where the file is located, and name as file name. We don't keep track of folders. By specifying the path to the folder of the file, describing folders become unnecessary. Only downside is that we are unable to handle empty folders as an empty folder is not described by any files.\\
\textbf{FileMetaData} holds meta date for a file such as resolution for pictures.\\
\textbf{FileInstance} is a relation between User and File. This describes the local path for a specific user, which allows different users to store their copy of a file, in different locations.\\
\textbf{Change} is used to keep track of who changed what in which file at what time.\\
\textbf{Project} holds the title of the project.\\
\textbf{ProjectHasFile} keeps track of which files a project references.\\
\textbf{UserHasProject} keeps track of which projects a user has.\\

\newpage